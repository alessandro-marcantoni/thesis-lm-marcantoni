\chapter{Introduction}
\markboth{INTRODUCTION}{INTRODUCTION}

In the context of a Pan-Europen project focused on defining the next generation of IoT with a human-in-the-loop approach, this thesis aims at expanding the vision for an accessible Integrated Development Environment that mixes Multi-Agent Oriented Programming and Hypermedia in a seamless interface for both humans and machines.

The thesis work was carried out while being hosted by the University of St.\ Gallen at the Interaction- and Communication-based Systems research group that is currently exploring the field of engineering autonomous systems.

The idea comes from the need to keep up with the fast digitalization of business activities and to provide a user-friendly tool that can be used to create and configure complex systems with low or no code.
Indeed, some efforts have already been made in this direction with the development of a block-based programming language for software agents.

However, the current state of the art does not provide a complete solution for the development of complex systems.
As a matter of fact, dealing with interactions and coordination between agents directly within them does not represent a scalable approach as the design complexity exponentially increases.

All Multi-Agent Systems possess some form of organization, although it may be implicit and informal.
With the increasing complexity of scenarios and the need to deal with a large number of agents, the need for an explicit specification of the organization at design time became more and more evident.
Nowadays, the organization is considered one of the first-class abstractions in the design of Multi-Agent Systems.

Therefore, this thesis aims at providing a solution that allows users to easily impose an organization on top of the agents.
Since ease of use and intuitiveness remain the key points for this project, users will be able to define organizations through the use of visual language and an intuitive development environment.

Chapter \ref{context} goes more in-depth into the motivations that brought the definition of this research proposal while Chapter \ref{background} provides a brief overview of the current state of the art for Multi-Agent Systems, Multi-Agent Oriented Programming, and Hypermedia Multi-Agent Systems.

In chapters \ref{requirements}, \ref{design}, and \ref{development} the development process is presented, focusing on the analysis of the problem and the reference technology, the design of the solution, and the implementation of the prototype.

Finally, Chapter \ref{chap:evaluation} describes the evaluation of the prototype that consisted of a test carried out with a group of users to gain qualitative feedback on the usability of the tool and to identify possible future improvements.