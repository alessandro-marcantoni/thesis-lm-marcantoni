\chapter{Evaluation}\label{chap:evaluation}
After developing a working prototype of the whole system, the next step was to prove that the solution satisfies the requirements.

This chapter describes the evaluation of the developed system.
In particular, the evaluation is divided into two parts:
\begin{enumerate}
    \item the first part involves the use of the developed system to solve a real-world problem;
    \item the second part is a qualitative evaluation of the developed visual language and development environment by users;
\end{enumerate}
As already mentioned, this twofold evaluation approach allows for assessing the user-friendliness of the development environment and the effectiveness of the visual language on one hand, and the expressivity and correctness of the latter on the other.

\section{Case Study - Smart Farming}
In order for users to have a real-world problem to solve, both during the design phase with the focus group and during the evaluation phase with the end users, a use-case scenario was developed.

Since the \textit{IntellIoT} project already defines three sectors of application, namely \textit{Agriculture}, \textit{Healthcare} and \textit{Manufacturing}, the most natural choice was to use one of them to frame the case study.
The \textit{Agriculture} sector was chosen for multiple reasons.
First of all, it is probably the most familiar to the research group thanks to recent projects.
Moreover, it involves concepts that are easy to understand even for non-experts and it is possibly the most diverse.

Therefore, the case study was developed around the \textit{Smart Farming} scenario where the aim is to define an organization for a typical farm that adopts IoT technologies.

\section{Outcome Analysis}