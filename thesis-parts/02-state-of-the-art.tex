\chapter{Background}\label{background}
To better understand this thesis work, a brief excursus of the existing literature about the main concepts the project relies on is presented. The aim is to provide some basic knowledge of the topics to understand what is considered state-of-the-art and to introduce some of the technologies exploited to develop the project.

\section{Multi-Agent Systems}
Multi-Agent Systems are considered a promising engineering style for developing adaptive software systems able to handle the continuous increase in their complexity.
Moreover, they allow the design and implementation of software systems using the same ideas and concepts that are the very founding of human societies and habits.

In this section, a brief overview of the main concepts and characteristics of MAS is provided, as well as the main approaches to their design.

\subsection{What is an Agent?}
The concept of a software agent can be traced back to the early days of research into Distributed AI in the 1970s when Carl Hewitt proposed the concurrent Actor model.
In his paper, he introduced the concept of a self-contained, interactive, and concurrently-executing object which he termed ``actor''.
The latter is a computational \textit{agent} with a mail address and behavior.
Actors can communicate by message-passing and carry out their actions concurrently.~\cite{hewitt1977viewing}

Software agents strongly rely on many of the concepts introduced by the Actor model, such as the idea of a self-contained entity endowed with its control flow that can interact with other entities and the use of message-passing for communication.
On top of that, agents bring in autonomous and proactive behavior as researchers were interested in the development of systems made up of entities with human-like skills such as reasoning, problem-solving, and decision-making.

The term ``agent'' quickly spread to heterogeneous research fields; therefore, there is no commonly agreed definition for it.
However, a generally accepted description of what an agent is is the following by Wooldridge~\cite{490039}:
\begin{quote}
    \textit{An agent is a self-contained program capable of controlling its own decision-making and acting, based on its perception of its environment, in pursuit of one or more objectives.}
\end{quote}
To sum up, a set of features that an agent should possess can be identified~\cite{490039}:
\begin{itemize}
    \item \textbf{Autonomy}: agents should be able to perform most of their tasks without the direct intervention of humans or other agents, and they should encapsulate control over their actions and internal state
    \item \textbf{Social ability}: agents should be able to interact with each other and possibly humans to complete their tasks.
    \item \textbf{Responsiveness} (situatedness): agents should perceive their environment and respond to changes in it.
    \item \textbf{Proactiveness}: agents should exhibit opportunistic, goal-directed behavior and take the initiative when appropriate.
\end{itemize}

Since the first years, the research concentrated on interaction and communication among agents, decomposition and distribution of tasks, and coordination and cooperation.
The goal was to specify, analyze, design, and integrate systems containing multiple collaborative agents.

\subsection{From the Individual to the Collective}
Multi-Agent Systems (MAS) have been studied as a per se field since the 1980s and gained widespread recognition in the 1990s.
Since then, international interest in the topic has grown enormously as agents are considered an appropriate paradigm to exploit the possibilities presented by massive open distributed systems.
Moreover, MAS seem to be a natural metaphor for understanding and building a wide range of what might be called \textit{artificial social systems}.~\cite{wooldridge2009introduction}

According to the \textit{Alan Turing Institute}~\cite{turing}
\begin{quote}
    \textit{A Multi-Agent System consists of multiple decision-making agents which interact in a shared environment to achieve common or conflicting goals.}
\end{quote}

\begin{figure}
    \centering
    \includegraphics[width=0.9\linewidth]{images/multi-agent-systems.png}
    \caption{A representation of Multi-Agent Systems.~\cite{jennings2000agent}}
    \label{fig:multi-agent-systems}
\end{figure}
Therefore, as it can also be noticed in \cref{fig:multi-agent-systems}, MAS are composed of an environment and agents existing within it that are bonded by relations.

\section{Multi-Agent Oriented Programming}
In principle, there is no constraint on the programming technologies that can be used to implement MAS.
However, the risk is to have an agent-centered interpretation of the system in which the environment and/or the organization contexts are represented and managed in the mind of the agents.
Therefore, the adoption of programming languages and paradigms that directly support first-class abstractions for these contexts highly simplifies the task of designing and implementing MAS and makes it possible to keep the level of abstraction coherent from design time to development time and finally also at runtime.

\textit{Multi-Agent Oriented Programming} (MAOP) is an approach to programming MAS that promotes the use of first-class programming abstractions that concern three main dimensions that characterize these systems:~\cite{boissier2020multi}
\begin{itemize}
    \item \textbf{Agent dimension}: it concerns the concepts and programming abstractions for the definition of the agents that participate in the system.
    Specifically, it allows the definition of software entities with their logical thread of control, which can autonomously and proactively achieve their goals, and interact with the environment, other agents, and the organization that regulates the system.
    \item \textbf{Environment dimension}: it offers concepts and abstractions to define the distributed resources and connections to the world shared among the agents.
    Thus, the environment abstraction is what makes the agents situated and provides them with tools that help them achieve their goals.
    \item \textbf{Organization dimension}: it collects all the concepts regarding the definition of relations, shared tasks, and policies among agents (inter)acting in a shared environment.
    In open systems, the organization is fundamental for coordination and regulation among agents.
\end{itemize}

Since agents have already been discussed, the following sections provide a description of the latter two dimensions, with a particular focus on the one regarding the organization, as it is the core of this thesis work.

\subsection{Environment in Multi-Agent Systems}
The environment in MAS plays a dual role~\cite{weyns2007environment}:
\begin{itemize}
    \item \textit{The ``external world''}:
    agents become aware of the context they are immersed in and its dynamics by perceiving the environment through sensors.
    Moreover, they pursue their goals through actions performed by actuators that aim at modifying the environment, eventually reaching the latter's desired state.
    \item \textit{A medium for coordination}:
    agents exploit the environment to share information and coordinate their behavior.
    Each agent follows simple behavioral rules, resulting in a collective behavior that is more complex than the sum of the individual behavior; this pattern resembles stigmergic systems in which agents coordinate their behavior through the manipulation of marks.
\end{itemize}
The environment not only enables the agents to interact with the deployment context, which they can access through sensors and actuators but also provides them with external resources that they can exploit to achieve their goal.

The \textit{Agents \& Artifacts} (A\&A) conceptual framework~\cite{ricci2007artifacts} argues that, just like in human society, MAS environments should contain different kinds of objects, tools, and artifacts in general that agents can use to support their activities.
This vision also constitutes a revolution from an engineering perspective as it encourages system designers to model the environment as a set of \textit{artifacts}, each of which encapsulates its intended purpose and exposes its observable state.
Moreover, the A\&A meta-model provides an effective abstraction level that shields low-level details of the deployment context, so that designers can focus on the agents' behavior.

\subsection{Organization in Multi-Agent Systems}
An agent organization can be defined as a social entity composed of a specific number of agents that accomplish several common tasks or goals and that are structured following some specific topology and communication interrelationships to achieve the main aim of the organization.
All MAS possess some form of organization, although it may be implicit and informal.

\subsubsection{Approaches to Organization in MAS}
There are two approaches to organizing agents in a MAS.~\cite{Ahmed_Abbas_2015}
The first one regards \textit{agent-centered} MAS, in which the focus is given to individual agents.
According to this viewpoint, the designer should concern about the local behavior of the agents and their interactions without worrying about the global structure and goal of the system as the latter should emerge as a result of the lower-level individual interactions in a bottom-up fashion.
The key issues of this approach are unpredictability and uncertainty since it could lead to undesirable emergent behaviors.
As Weyns~\cite{weyns2010organizations} stated, giving the responsibility of system organization implicitly to individual agents is highly complex and not suitable for real-world large-scale scenarios.

The second approach is \textit{organization-centered} MAS, in which the structure of the system is given greater attention.
The developer designs the entire organization and coordination patterns on the one hand, and the agents' local behavior on the other.
This can be seen as a top-down approach as the organization abstractions impose constraints on the agents and regulate their interactions, simplifying the design of complex and scalable systems and allowing more accurate modeling of the problems being tackled.
On top of that, the organization-centered approach avoids the emergence of undesirable behaviors such as divergence.
Indeed, larger MAS bring a higher risk of divergence, therefore the need for explicit organizational regulation.

\subsubsection{Organizational Paradigms}
Although no two organizational instances are likely to be identical, there are identifiable classes of organizations, that emerged from research and real-world applications, which share common characteristics.
These classes are called \textit{organizational paradigms} and cover particularly common, useful, or interesting structures that can be described in some general form.
Here is provided a brief overview of the most common paradigms~\cite{horling_lesser_2004}:
\begin{itemize}
    \item \textbf{Hierarchies}: agents are conceptually arranged in a tree-like structure, where agents higher in the tree have a more global view than those below them.
    Interactions only take place between connected entities; data produced by lower-level agents in a hierarchy typically travels upwards to provide a broader view, while control flows downwards as the higher-level agents provide directions to those below.
    \item \textbf{Holarchies}: agents are organized in holons.
    The term \textit{holon} comes from the Greek words \textit{holos}, meaning ``whole'', and \textit{on}, meaning ``part''. Therefore, a holon is a self-contained entity that can be considered both as a part of a larger entity and as a whole in itself, and that has a character derived but distinct from the entities that make it up and at the same time it contributes to the properties of a greater whole.
    Examples showed how holarchies can be used to effectively model the division of labor in MAS, where capabilities and tasks were imparted to holons instead of single agents.
    This results in a layer of abstraction that allows other entities to interact with a group as a single functional unit.
    \item \textbf{Coalitions}: they are formed by agents that share a common goal and that are willing to cooperate to achieve it and are generally short-lived as they are formed with a purpose in mind and dissolved when that need no longer exists.
    Although there may be a distinguished ``leading agent'', within a coalition the structure is typically flat.
    However, since once formed coalitions may be treated as a single entity, it is possible to form a hierarchical structure by nesting coalitions.
    \item \textbf{Teams}: they consist of several cooperative agents that agreed to work together towards a common goal.
    Unlike coalitions, teams attempt to maximize the performance of the group as a whole, rather than the performance of individual agents.
    This is usually achieved by assigning roles to the agents, which become responsible for specific tasks, and by providing the agents with representations of the shared goals, knowledge, and plans.
    \item \textbf{Congregations}: although similar to the latter two structures, they differentiate because they are assumed to be long-lived and are not necessarily formed with a specific goal in mind.
    Indeed, congregations are formed among agents with similar o complementary characteristics to facilitate the process of finding partners for collaborations.
    \item \textbf{Societies}: they are open systems where agents of different kinds may come and go at will while the society persists, acting as an environment through which the participants meet and interact.
    Societies impose on agents a set of constraints which are known as \textit{social laws}, \textit{norms}, or \textit{conventions}.
    These represent rules by which agents must act and provide a level of consistency in behavior that facilitates the coexistence of possibly heterogeneous agents.
    \item \textbf{Federations}: they are groups of agents which have ceded some amount of autonomy to a single delegate who represents the group.
    The members of the group interact only with the delegate, who accepts skills and needs descriptions from them, which are then used to match with requests from delegates representing other groups.
    \item \textbf{Markets}: buying agents may request or place bids for a common set of items, such as shared resources, tasks, services, or goods, or even supply items to the markets to be sold.
    On the other hand, sellers are responsible for processing bids and determining the winner.
    \item \textbf{Matrix}: they can be seen as a relaxation of the one-agent, one-manager restriction in hierarchical organizations, that permit many managers to influence the activities of an agent.
\end{itemize}

\begin{figure}
    \begin{subfigure}[h]{0.3\linewidth}
        \includegraphics[width=\textwidth]{images/orgs/org-hierarchy.png}
        \caption{Hierarchiy}
        \label{fig:hierarchy}
    \end{subfigure}
    \begin{subfigure}[h]{0.3\linewidth}
        \includegraphics[width=\textwidth]{images/orgs/org-holarchy.png}
        \caption{Holarchy}
        \label{fig:holarchy}
    \end{subfigure}
    \begin{subfigure}[h]{0.3\linewidth}
        \includegraphics[width=\textwidth]{images/orgs/org-coalitions.png}
        \caption{Coalitions}
        \label{fig:coalitions}
    \end{subfigure}
    \begin{subfigure}[h]{0.3\linewidth}
        \includegraphics[width=\textwidth]{images/orgs/org-teams.png}
        \caption{Team}
        \label{fig:teams}
    \end{subfigure}
    \begin{subfigure}[h]{0.25\linewidth}
        \includegraphics[width=\textwidth]{images/orgs/org-congregations.png}
        \caption{Congregations}
        \label{fig:congregations}
    \end{subfigure}
    \begin{subfigure}[h]{0.3\linewidth}
        \includegraphics[width=\textwidth]{images/orgs/org-societies.png}
        \caption{Society}
        \label{fig:societies}
    \end{subfigure}
    \begin{subfigure}[h]{0.3\linewidth}
        \includegraphics[width=\textwidth]{images/orgs/org-federations.png}
        \caption{Federations}
        \label{fig:federations}
    \end{subfigure}
    \begin{subfigure}[h]{0.3\linewidth}
        \includegraphics[width=\textwidth]{images/orgs/org-markets.png}
        \caption{Markets}
        \label{fig:markets}
    \end{subfigure}
    \begin{subfigure}[h]{0.3\linewidth}
        \includegraphics[width=\textwidth]{images/orgs/org-matrix.png}
        \caption{Matrix}
        \label{fig:matrix}
    \end{subfigure}
    \caption{Visual representation of the organizational paradigms.}
    \label{fig:organizational-paradigms}
\end{figure}

All of the above structures come with their benefits and drawbacks
%, which are briefly summarized in~\cref{table:orgs-advantages-disadvantages}
and it is generally agreed that there is no single type of organization that is suitable for all situations.
Indeed, sometimes two or more organizational paradigms may be combined to form a compound organization, exploiting features of each of the component organizations.


% \begin{table}[h]
%     \centering
%     \begin{tabular}{ | l p{0.4\linewidth} p{0.4\linewidth} | }
%         \hline
%         \textbf{Paradigm} & \textbf{Benefits} & \textbf{Drawbacks} \\\hline\hline
%         Hierarchy & Maps to many common domains; handles scale well & Potentially brittle; can lead to bottlenecks \\\hline
%         Holarchy & Exploit autonomy of functional units & Must organize holons, lack of predictable performance. \\\hline
%         Coalition & Exploit strength in number & Individual goals are preferred to common goals \\\hline
%         Team & Address larger problems; task-centric & Increased communication \\\hline
%         Congregation & Facilitates agents discovery & Sets may be overly restrictive \\\hline
%         Society & Public services; well-defined conventions & Potentially complex; agents may require society-related capabilities \\\hline
%         Federation & Matchmaking, brokering, and translation services facilitate dynamic agent pool & Intermediaries become bottlenecks \\\hline
%         Market & Good at allocation, increased utility through centralization, increased fairness through bidding & Potential for malicious behavior; decisions complexity can be high \\\hline
%         Matrix & Resource sharing; multiple influenced agents & Potential conflicts \\\hline
%     \end{tabular}
% \caption{Benefits and drawbacks of the organizational paradigms, adopted from~\cite{Ahmed_Abbas_2015}.}
% \label{table:orgs-advantages-disadvantages}
% \end{table}

\subsection{The JaCaMo Platform}
Regarding the engineering and implementation of MAS, one of the reference technologies is the JaCaMo platform~\cite{Boissier2016}, which supports practical programming based on the first-class abstractions introduced before to develop organized agents situated in a shared environment.
Therefore, the platform gives convenient tools to program agents, their environment, and the organizations they belong to.
JaCaMo is built on top of three other existing platforms:
\begin{itemize}
    \item \jason{}: provides a programming language to code autonomous intelligent agents based on the BDI (Belief, Desire, Intention) architecture~\cite{bordini2007programming}\cite{Bratman1987-BRAIPA}.
    \item \cartago{}: as the way to define the environment in which agents will be situated, following the \textit{A\&A} metamodel~\cite{Ricci2009}.
    \item \moise{}: based on the \moiseplus{} model~\cite{10.1145/544741.544858}, it allows the explicit definition and management of organizations within the systems~\cite{doi:10.1504/IJAOSE.2007.016266}.
\end{itemize}

Since the JaCaMo platform was chosen as the enabling tool supporting the instantiation of the organizations built with the visual language developed, the latter takes inspiration from the \moiseplus{} model and its concepts, which are briefly described in the following section.

\subsubsection{The \moiseplus{} Model}
Just like the \moise{} model~\cite{hannoun2000}, which it extends, the \moiseplus{} model aims at providing a way to cope with both the agent-centered and organization-centered approaches to the design of MAS.
This way, the ability to manage the complexity taken from the organization-centered approach, and the flexibility taken from the agent-centered approach, can be combined to face the constantly growing complexity of MAS applications.
Indeed, the MAS designer can specify an \textit{organization specification} (OS), which defines an ``a priori'' set of constraints and cooperation patterns imposed on the agents; on the other hand, agents themselves can reason about and modify the \textit{organization entity} (OE), which is the actual instantiation of the organization on the agents.

In addition, the \moiseplus{} model proposes an organizational modeling language that explicitly decomposes the specification of organizations into \textit{structural}, \textit{functional}, and \textit{deontic} dimensions~\cite{doi:10.1504/IJAOSE.2007.016266}\cite{10.1145/544741.544858}.

The structural dimension specifies the \textit{roles}, \textit{groups}, and \textit{links} of the organization.
The definition of roles states that when agents decide to play a role, they are accepting some behavioral constraints related to the role.

The functional dimension specifies how the \textit{global collective goals} should be achieved.
Specifically, how these goals are decomposed in \textit{plans}, grouped in coherent sets by \textit{missions}, and how are distributed to the agents.
The decomposition of global goals results in a goal tree, called \textit{scheme}, where the leaf goals can be achieved directly by the agents.

Finally, the deontic dimension serves as a binding between the structural and functional dimensions, specifying the roles' \textit{permissions} and \textit{obligations} for missions.

The infrastructure adopted by JaCaMo for the \moiseplus{} model is called \oraformas{} (Organizational Artifacts for Multi-Agent Systems)~\cite{hubner2010} which conceives \textit{organizational agents} that control, manage and adapt the organization operating on \textit{organizational artifacts}, thus adhering to the \textit{A\&A} metamodel, such as \textsf{OrgBoard}, \textsf{GroupBoard}, \textsf{SchemeBoard}, and \textsf{NormativeBoard}.

The next section illustrates how the above artifacts can be used in the deployment of an organization.

\begin{figure}[H]
    \centering
    \includegraphics[width=\linewidth]{images/org-artifacts.png}
    \caption{Organizational artifacts in \oraformas{} with their interface including operations, observable properties, and link interfaces. Adopted from~\cite{hubner2010}.}
    \label{fig:org-artifacts}
\end{figure}

\subsubsection{\oraformas{} in Action}
When a set of agents wants to coordinate their actions to achieve a common goal, they can do it, for instance, through the following steps:
\begin{enumerate}
    \item One of the agents, which is also an \textit{organizational agent}, creates an \textsf{OrgBoard} artifact based on a specification.
    \item The organizational agent creates a \textsf{GroupBoard} artifact for each group of agents that will be part of the organization.
    Once the \textsf{GroupBoard} is created,  the artifact registers itself in the \textsf{OrgBoard} exploiting the latter's link interface.
    \item All the other agents get notified about the new artifacts and therefore may decide to adopt a role in one of the groups.
    They can do so using the \textsf{adoptRole} operation of the \textsf{GroupBoard} artifact.
    \item The organizational agent can now create a \textsf{SchemeBoard} artifact to start the organization's collective goals.
    As for the \textsf{GroupBoard}, the \textsf{SchemeBoard} artifact registers itself in the \textsf{OrgBoard}.
    As every \textsf{SchemeBoard} has one \textsf{NormativeBoard}, the latter is created automatically and linked to the former.
    \item Once the \textsf{SchemeBoard} is created, obligations and permissions are computed and verified by the \textsf{NormativeBoard}.
    Agents can now commit to their missions according to the \textsf{NormativeBoard} rules.
    \item Once the scheme is well formed, the goals of the scheme can be achieved by the agents.
\end{enumerate}

\section{Hypermedia Multi-Agent Systems}
The current technological landscape brings new challenges to the engineering of MAS such as the support of large-scale open systems, and the support of heterogeneous agents and humans in the loop.

\subsection{The World Wide Web}
The Web has had remarkable success as a worldwide and long-lived system of people, providing them with a \textit{distributed hypermedia environment}, composed of interrelated Web pages, that they can navigate and use in pursuit of their goals.

No doubt REST (REpresentational State Transfer), the architectural style of the Web~\cite{fielding2000}, is one of the enabling factors of the above characteristics.
REST consists of a set of architectural constraints, such as \textit{client-server} and \textit{stateless} interaction, and \textit{uniform interface}~\cite{fielding2002}.
The latter principle is fundamental for RESTful systems, as it simplifies and decouples the Web architecture, and it is achieved through four constraints:
\begin{itemize}
    \item \textit{Identification of resources}: each Web-based concept is modeled as a resource identified by a URI.
    \item \textit{Manipulation of resources through representations}: clients manipulate representations of resources.
    The same resource can be represented in different ways, e.g. as HTML, XML, or JSON.
    The key point is that the representation is a way to interact with a resource but it is not the resource itself.
    \item \textit{Self-descriptive messages}: each message includes enough information to describe how to process the message.
    \item \textit{Hypermedia as the engine of application state (HATEOAS)}: the representation of a resource includes links that the client can use to dynamically discover other resources, therefore enabling \textit{hypermedia-driven interaction}~\cite{Varanasi2015}.
\end{itemize}

According to HATEOAS, a typical Web application, which is composed of multiple hyperlinked Web pages, can be seen as a finite state machine where each page represents a state and hyperlinks between pages represent transitions between states.
Indeed, given the URI of a page, a client can dereference the URI to retrieve an HTML representation of that page.
This action triggers a transition to a new application state which provides the client with a new set of reachable states in the form of hyperlinks to other Web pages.
Similarly, the client can send a request to the server to update a resource, thus triggering a transition to a new state.

The key point is that both the next reachable states and the knowledge required to transition to those states are conveyed to the client through hypermedia.
Therefore, a client should be able to discover new resources and how to use them at runtime, allowing components to be deployed independently from one another.

\subsection{The Web of Things}
The Web of Things (WoT)~\cite{wot}, first introduced in 2007~\cite{guinard2011web}, is a set of W3C standards that aim to improve the interoperability and usability of the Internet of Things (IoT).
The idea is to apply the architectural principles and standardized technologies of the Web to integrate the different technological stacks used by the current IoT \textit{things}.

One of the fundamental building blocks of the WoT is the \textit{Thing Description (TD)}~\cite{wottd} that acts as a machine-readable manual for the interaction with the \textit{thing} it describes.
The TD is based on the concept of \textit{interaction affordances} which refer to the perceived and actual properties of the \textit{thing} that determine how the latter can be used.
The three types of affordances are:
\begin{enumerate}
    \item \textbf{Properties}: they represent the state of the \textit{thing} that can be read and/or written.
    \item \textbf{Actions}: they allow the invocation of a function of the \textit{thing} which manipulates its state or triggers a process.
    \item \textbf{Events}: they describe an event source that asynchronously pushes event data to the observers.
\end{enumerate}
The affordances of a \textit{thing} are intended in the hypermedia perspective of presenting information and control, suggesting to the clients the possible choices for interaction and how to use them in the form of hyperlinks.

\subsection{Web-based Multi-Agent Systems}
There has been extensive research on using the Web as an infrastructure for distributed MAS.
The early approaches usually fall into one of the following two categories:
\begin{itemize}
    \item \textbf{The Web as a Transport Layer}: these systems use HTTP as a transport layer for the communication between agents; thus, they make limited use of the architectural properties of the Web.
    \item \textbf{The Web as a non-Hypermedia Application Layer}: agent services are translated into Web services, which expose REST-like interfaces.
    Even tho these systems typically respect the first three uniform interface constraints, they do not support HATEOAS and therefore hypermedia-driven interaction, making clients and servers tightly coupled to one another, which is an important limitation when engineering large-scale, open systems.
\end{itemize}

The premature development of these approaches, not completely adhering to the Web architectural style, and the lack of crucial initiatives such as the Web of Things, together with the shortage of real-world applications, have hindered their widespread acceptance.

\subsection{The Bridge between the Web and Multi-Agent Systems}

\begin{figure}
    \centering
    \includegraphics[width=\linewidth]{images/hypermedia-mas.png}
    \caption{Hypermedia MAS. Adopted from~\cite{ciortea2019}.}
    \label{fig:hypermedia-mas}
\end{figure}

Integrating the environment in multi-agent systems with the Web architecture helps bridge the gap that previously existed between MAS and the Web~\cite{ciortea2019}.
Hypermedia MAS are systems composed of both people and autonomous agents situated in a shared \textit{hypermedia environment} that is distributed across the Web and thus becomes an \textit{hypermedia application}~\cite{ciortea2018}.
According to this approach, all the entities, both agents and artifacts, are Web resources and their representations can be related by hyperlinks.

In contrast to typical environments, a hypermedia environment uses hypermedia to drive interaction in the MAS: agents navigate and crawl the environment at runtime to discover other entities in the MAS, as well as the means to interact with them, in an analogous approach to the one used in the Web of Things through the Thing Description.
This reduces coupling and enhances the scalability and evolvability of the systems.

The three key design principles meant to ensure the proper use of hypermedia as a general mechanism for uniform interaction in MAS are the following~\cite{10.1007/978-3-030-25693-7_15}:
\begin{enumerate}
    \item \textbf{Uniform resource space}: all entities in a hypermedia MAS and relations among them should be represented in the hypermedia environment in a uniform, resource-oriented manner.
    For instance, one agent could send a message to another by writing an RDF representation of the message in the hypermedia.
    To receive messages, an agent could observe a resource that represents its mailbox in the hypermedia.
    To turn on a light, an agent could manipulate the state of a resource that represents the light bulb in the hypermedia.
    Anyways, interactions between agents and resources in their hypermedia environment should conform to the REST constraints.
    \item \textbf{Single entry point}: given a single entry point into the environment of a hypermedia MAS, an agent should be able to discover the knowledge required to participate in the system by navigating the hypermedia.
    The core idea is to minimize coupling by enabling system-wide discoverability as agents can crawl the hypermedia to discover what other agents, tools, or entities in the system can help them achieve their goals.
    Equally important, agents can also discover in the hypermedia how to interact with entities through their affordances.
    \item \textbf{Observability}: in a hypermedia MAS, any resource in the hypermedia environment that could be of interest to agents should be observable.
    While the first two principles ensure the dynamic discovery of a hypermedia MAS via crawling, the latter promotes the use of mechanisms that allow agents to selectively observe entities of interest.
    This is important to improve the scalability and handle larger environments.
\end{enumerate}

Engineers can choose to ignore one or more of these principles, but the MAS would most likely make limited use of the hypermedia and would not achieve uniform interaction, hindering the scalability and openness of the system.

% semantic web
% visual programming
% agent-oriented visual programming